\documentclass{article}
\usepackage[utf8]{inputenc}

\title{Regresja Liniowa - przebieg}
\author{Krzysztof Czuba, Grzegorz Mazur, Julia Zapora}
\date{March 2022}

\begin{document}

\maketitle
\section{dane wejściowe}

\begin{center}\large{
\begin{tabular}{ c c c c c c }
    $H$ & $t_{1}$ & $t_{2}$ & $t_{3}$ & $t_{4}$ & $t_{5}$ \\
	0,4 & 0,291 & 0,290 & 0,291 & 0,291 & 0,291 \\
	0,5 & 0,324 & 0,324 & 0,324 & 0,325 & 0,325 \\
	0,6 & 0,355 & 0,355 & 0,355 & 0,355 & 0,355 \\
	0,7 & 0,383 & 0,382 & 0,382 & 0,382 & 0,382 \\
	0,8 & 0,408 & 0,408 & 0,408 & 0,408 & 0,408 \\
	0,9 & 0,434 & 0,434 & 0,433 & 0,433 & 0,433 \\
	1,0 & 0,456 & 0,456 & 0,456 & 0,456 & 0,456 \\
    1,1 & 0,478 & 0,478 & 0,478 & 0,479 & 0,479 \\
	1,2 & 0,498 & 0,498 & 0,498 & 0,498 & 0,498 \\
	1,3 & 0,518 & 0,518 & 0,518 & 0,519 & 0,518 
\end{tabular}
}\end{center}

\begin{center}\large{
    $H$ - wysokość [$m$] \\
    $t_{i}$ - pomiar dla danej wysokości [$s$] \\
    wszystkie wartości przybliżone do trzeciego miejsca po przecinku
}\end{center}

\newpage
\section{dane do regresji}
\begin{center}\large{
\begin{tabular}{ c c }
    $\sqrt{H}$ & $\bar{t}$\\
	0,632 & 0,291 \\
	0,707 & 0,324 \\
	0,775 & 0,355 \\
	0,834 & 0,382 \\
	0,894 & 0,408 \\
	0,949 & 0,433 \\
	1,000 & 0,456 \\
    1,049 & 0,478 \\
	1,095 & 0,498 \\
	1,140 & 0,518
\end{tabular}}
\end{center}

\begin{center}\large{
    $\sqrt{H}$ - pierwiastek wysokości [$m$] \\
    $\bar{t}$ - średnia pomiarów dla danej wysokości [$s$] \\
    wszystkie wartości przybliżone do trzeciego miejsca po przecinku
}\end{center}

\newpage
\section{obliczenia do regresji}
\begin{center}\huge{
    $
    N = 10 \\ ~ \\
    S_{x} = \Sigma_{i=1}^{N} \sqrt{H}_{i} \\ ~ \\
    \small{
    0,632 + 0,707 + 0,775 + 0,834 + 0,894 + 0,949 + 1,000 + 1,049 + 1,095 + 1,140 =\\
    }

    \large{= 9.078 = S_{x}} \\ ~ \\
    \hline \\ ~ \\
    \huge{
    S_{y} = \Sigma_{i=1}^{N} \bar{t_{i}} \\ ~ \\
    \small{
    0,291 + 0,324 + 0,355 + 0,382 + 0,408 + 0,433 + 0,456 + 0,478 + 0,498 + 0,518 =\\ 
    \large{= 4,144 = S_{y}} 
    } \\ ~ \\
    \hline \\ ~ \\
    \huge{
    S_{xx} = \Sigma_{i=1}^{N} \sqrt{H}_{i}^{2} \\ ~ \\
    }
    \small{
        0,632^{2} + 0,707^{2} + 0,775^{2} + 0,834^{2} + 0,894^{2} + 0,949^{2} + 1,000^{2} + 1,049^{2} + 1,095^{2} + 1,140^{2} =\\
    }
    \large{= 8,500 = S_{xx}} \\ ~ \\
    \hline \\ ~ \\
    \huge{
    S_{xy} = \Sigma_{i=1}^{N} \sqrt{H}_{i} * \bar{t_{i}} \\ ~ \\
    }	
    \small{
    0,632 * 0,291 + 0,707 * 0,324 + 0,775 * 0,355 +
	0,834 * 0,382 + 0,894 * 0,408 + 0,949 * 0,433 +
	1,000 * 0,456 + 1,049 * 0,478 + 1,095 * 0,498 +
	1,140 * 0,518 = \\
    }\large{= 3,878 = S_{xy}}
    }$}
\end{center}
\newpage
\begin{center}\huge{$
    a = \frac{N * S_{xy} - S_{x} * S_{y}}{N * S_{xx} - S_{x} * S_{x}} = 0,448
    $} \\ ~ \\
    \huge{$
    b = \frac{S_{xx} * S_{y} - S_{x} * S_{xy}}{N * S_{xx} - S_{x} * S_{x}} = 0,007
    $}\\ ~ \\
    \large{Wszystkie wartości przybliżone do trzeciego miejsca po przecinku}
\end{center}

\end{document}
